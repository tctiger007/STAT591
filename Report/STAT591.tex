\documentclass[11pt]{report}
\usepackage{./assignment}
\usepackage{slashbox}
\usepackage{graphicx}
\usepackage{subfigure}
\usepackage{enumerate}
\usepackage[shortlabels]{enumitem}
\usepackage{stmaryrd}
\usepackage[final]{pdfpages}
\usepackage{array}
\usepackage{multirow}
\usepackage[T1]{fontenc}
\usepackage[utf8]{inputenc}
\usepackage{authblk}
\usepackage{amsmath}
\usepackage{amssymb}
\usepackage{epstopdf}

\input{./Definitions}

\begin{document}
\title{
  \huge STAT 591 Summary Report \\ 
  \vspace{10mm}
  \large  Functional Data Analysis for Sparse Longitudinal Data\\
  \normalsize Fang Yao, Hans-Georg M\"{u}ller \& Jane-Ling Wang}

\author{Wangfei Wang \\ wwang75@uic.edu }

\graphicspath{./Figures/}

\maketitle

\section{INTRODUCTION}

Functional principal components (FPC) analysis can reduce random trajectories of random curves to a set of FPC scores, and therefore is popular in longitudinal data analysis. 
FPC analysis characterizes the dominant mode of variation of 
In longitudinal data analysis, it is not uncommon that repeated measurements are infrequent, and these measurements are irregularly spaced per subject. 
In this situation, FPC analysis has limitations because of the sparse repeated measurements. 


To overcome this, the authors of the paper proposed a version of functional principal components (FPC) analysis, in which they framed the FPC scores as conditional expectations. 
And thus they coined this method ``principal components analysis through conditional expectation (PACE)''.


But the current limitation of using FPC analysis in longitudinal data is that when the longitudinal data have sparse repeated measures, it 

\section{FUNCTIONAL PRINCIPAL COMPONENTS ANALYSIS FOR SPARSE DATA}

\subsection*{Model with Measurement Errors}
\begin{align}
	\label{eq:eq1}
	Y_{ij} &= X_i(T_{ij}) + \epsilon_{ij}  \\
	&= \mu(T_{ij}) + \sum_{k=1}^\infty \xi_{ik}\phi_k(T_{ij}) + \epsilon_{ij}, \; \; \;  T_{ij} \in \mathcal{T} 
\end{align}
where $\expec{\epsilon_{ij}} = 0$, var($\epsilon_{ij}$) = $\sigma^2$


\subsection*{Estimation of the Model Components}
In equation \eqref{eq:eq1}, $cov(Y_{ij}, Y_{il} = cov(X(T_{ij}, X(T))))$

\subsection*{Functional Principal Components Analysis Through Conditional Expectation}

\subsection*{Asymptotic Confidence Bands for Individual Trajectories}

\section{ASYMPTOTIC PROPERTIES}

\section{SIMULATION STUDIES}

\section{APPLICATIONS}

\subsection*{Longitudinal CD4 Counts}

\subsection*{Yeast Cell Cycle Gene Expression Profiles}





\end{document}
